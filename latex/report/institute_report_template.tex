% !TeX spellcheck = pl_PL
% !TEX encoding = utf8

% Wspólne katalogi.
\def\ROOTPATH{../../../..}
\def\COMDIR{\ROOTPATH/common}
\def\PICS{\ROOTPATH/rysunki}
%\def\ROBREXPICS{\ROOTPATH/rysunki/TK/2013_robrex_specyfikacja/}
%\def\SORTERPICS{\ROOTPATH/rysunki/TK/2013_sorter/}

%%%%%%%%%%%%%%%%%%%%%%%%%%%%%%%%%%%%%%%%%%%%%%%%%%%%%%%%%%%%%%%%%%%%%%%%%
% Klasa dokumentu.
\documentclass{../common/tk_iaiis}

%%%%%%%%%%%%%%%%%%%%%%%%%%%%%%%%%%%%%%%%%%%%%%%%%%%%%%%%%%%%%%%%%%%%%%%%%
% Pakiety.
\usepackage{lmodern}
\usepackage{subfig}
\usepackage{\COMDIR/agent-symbols}
\usepackage{amssymb}

% importowanie obrazków pdf+tex z Inkscape
\usepackage{import}


%%%%%%%%%%%%%%%%%%%%%%%%%%%%%%%%%%%%%%%%%%%%%%%%%%%%%%%%%%%%%%%%%%%%%%%%%
% Przydatne makra.
\def\rys#1{rys.~\ref{fig:#1}}
\def\Rys#1{Rys.~\ref{fig:#1}}
\newcommand{\ang}[1]{ang. \textit{#1}}

% Czerwony komentarz TK.
\usepackage{color}
\newcommand{\ckk}[1]{\textcolor{red}{TK: #1}}
\newcommand{\csk}[1]{\textcolor{red}{\footnote{\ckk{#1}}}}
\newcommand{\cik}[2]{\ckk{#1 :\begin{itemize}#2\end{itemize}}}

% Niebieski komentarz Karola.
\newcommand{\nkk}[1]{\textcolor{blue}{KK: #1}}
\newcommand{\nik}[2]{\nkk{#1 :\begin{itemize}#2\end{itemize}}}


%%%%%%%%%%%%%%%%%%%%%%%%%%%%%%%%%%%%%%%%%%%%%%%%%%%%%%%%%%%%%%%%%%%%%%%%%
% Strona tytułowa.
\title{}
\period{01.09.2013 -- 30.11.2013}
\author{}
\date{\today}
\documenttype{Raport Instytutowy nr XXX}
\documentversion{draft}




%%%%%%%%%%%%%%%%%%%%%%%%%%%%%%%%%%%%%%%%%%%%%%%%%%%%%%%%%%%%%%%%%%%%%%%%%
\begin{document}

% Strona tytułowa.
\maketitle

% Spis treści.
\tableofcontents

\chapter{Wstęp}
\cik{Ten raport ma opisać wykorzystanie wnioskowania probabilistycznego w zadaniu rozróżniania obiektów płaskich od wypukłych}
{
\item kilka słów o aktywnej wizji (taka mikroprzeglądówka), 
\item o zadaniu,
\item o poprzednich wersjach (odnośniki do KKR/MMAR),
\item o tym, że opracowano strukturę systemu oraz zachowania i prosty automat realizujący zadanie
\item wyjawienie celu prac: model + wnioskowanie
}


W Rozdziale \ref{sec:implementation} opisane zostały wykorzystane narzędzia programistyczne, z naciskiem na nowowytworzone komponenty oraz zadanie służące do rozróżniania obiektów płaskich od wypukłych/trójwymiarowych.

\chapter{Model obiektu oraz proces wnioskowania}
\nik{Tutaj należy opisać}
{
\item kilka słów o pgm, bn, rodzaje wnioskowania
\item model sieci bayesa dla niebieskiej piłeczki (a raczej dwa), w tym opis struktury sieci, wagi
}


\bibliographystyle{plabbrv}
\bibliography{\COMDIR/robot}

\end{document}
